\documentclass[11pt]{letter}
\usepackage{ebgaramond}
\usepackage{geometry}
\usepackage[super]{nth}
\usepackage[dvipsnames]{xcolor}
\geometry{
	a4paper,
	total={170mm,257mm},
	left=40mm,
	right=40mm,
	bottom=20mm,
	top=20mm,
}
\usepackage{hyperref}

\hypersetup{
	colorlinks=true,
	linkcolor=black,
	citecolor=black,
	filecolor=black,
	urlcolor=black,
}

\begin{document}
% If you want headings on subsequent pages,
% remove the ``%'' on the next line:
% \pagestyle{headings}

\begin{letter}{}
\address{Dr. Natalie B. Hogg,\\ Institute of Astronomy,\\
University of Cambridge,\\
Madingley Road, Cambridge, UK.\\
Email: nbh25@cam.ac.uk}

\opening{Dear Prof. Hook, Prof. Nichol and Dr. Nightingale,}

I am writing to request transfer of my Euclid Consortium membership from Euclid France to Euclid UK. This is because as of \nth{6} October I have moved from France to the UK and am now affiliated with a UK institution. Previously I was a postdoc at the Laboratoire univers et particules, a mixed research unit attached to IN2P3/CNRS and the University of Montpellier; I am now a postdoc at the Institute of Astronomy, University of Cambridge.

Within Euclid, I am currently active in the Strong Lensing Science Working Group, having participated in visual inspection tasks for the ERO and Q1 data, as well as making minor contributions to some of the associated papers. Beyond strong lens discovery and classification, I intend to eventually lead a key project to measure the correlation between weak lensing and strong lensing shear in Euclid data, however, there are both theoretical and computational challenges, particularly surrounding strong lens modelling and parameter inference, to be overcome before this measurement can be attempted. The latter is a key focus of my new postdoc position.

I am also currently contributing to the DR1-KP-TH2 project, which is investigating potential departures from the distance-duality relation in Euclid DR1 data. This relation states that luminosity and angular diameter distances must coincide under certain assumptions, the violation of which may indicate the presence of new physics. Within this project, I have been contributing to writing and reviewing the code we will use for the analysis, and will participate in the data analysis and paper writing at the appropriate time.

Thank you for your consideration in this matter.

%I am currently a postdoctoral researcher at the Laboratoire Univers et Particules (LUPM) at the Universit\'e de Montpellier, working with Prof. Julien Larena and Dr. Pierre Fleury in the field of gravitational lensing and its application to cosmology. In particular, our team is pioneering the study of the so-called line-of-sight shear -- a robust parameterisation of the effect of weak lensing on strong lensing images which has the potential to provide independent constraints on cosmological parameters such as $\sigma_8$.
%
%The Euclid mission is expected to observe tens of thousands of strong gravitational lenses, an ideal dataset from which to measure the line-of-sight shear and obtain high precision cosmological constraints. I have expertise in both the theoretical aspects of strong lensing and in modelling strong lensing images with statistical techniques such as MCMC sampling. In particular, I led the work which demonstrated that the line-of-sight shear is measureable using mock images (\href{https://arxiv.org/abs/2310.11977}{Hogg et al. 2023}) and I am currently working on modelling SLACS strong lenses to make the first measurement of this quantity in real data. 
%
%I am also an active participant in the Lensing Working Group of the COSMOS-Web collaboration, a JWST Cycle 1 Treasury program obtaining near-infrared imaging of the COSMOS field. In this group, I have contributed to forecasting the expected number of strong lenses, identifying lens candidates in the raw data, and will soon start to work on modelling the first confirmed strong lens candidates. I expect that my contributions to the Strong Lensing Science Working Group in Euclid will be of a similar nature.

\signature{Dr. Natalie Hogg}

\closing{Yours sincerely,}

%enclosure listing
%\encl{}

\end{letter}
\end{document}
