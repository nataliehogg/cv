\documentclass[11pt]{letter}
\usepackage{ebgaramond}
\usepackage{geometry}
\geometry{
	a4paper,
	total={170mm,257mm},
	left=40mm,
	right=40mm,
	bottom=20mm,
	top=20mm,
}

\begin{document}
% If you want headings on subsequent pages,
% remove the ``%'' on the next line:
% \pagestyle{headings}

\begin{letter}{}
\address{Dr. Natalie B. Hogg,\\ Laboratoire Universe et Particules,\\
Universit\'e de Montpellier,\\
Montpellier, France, 34070.\\
Email: natalie.hogg@lupm.in2p3.fr}

\opening{Dear Dr. Starck and Dr Bournaud,}

I am writing to apply for the permanent research position in cosmology at CosmoStat. This position was brought to my attention by my colleague Dr. Vivian Poulin. I am currently a postdoctoral researcher at the Laboratoire Univers et Particules (LUPM) at the Universit\'e de Montpellier, working with Prof. Julien Larena and Dr. Pierre Fleury in the field of gravitational lensing and its application in cosmology.

During my PhD (University of Portsmouth, 2021), I led two projects on the development and analysis of alternative dark energy models, showing how, from a statistical standpoint, these models are disfavoured by the data compared to $\Lambda$CDM. During the course of these projects, I became an expert in the use of Boltzmann codes and MCMC parameter inference methods. Alongside my thesis work, I also led a project on forecasting the constraints on violations of the distance duality relation that future standard siren observations will provide.

As a postdoc, I have successfully applied my analysis and programming skills to the field of gravitational lensing, and specifically the interface between the strong and weak lensing regimes. I led the implementation of a new formalism to describe the weak lensing of strong lensing (also known as ``line-of-sight effects'') in the public strong lensing software \texttt{lenstronomy}, becoming one of the top three contributors to the code. Furthermore, I have demonstrated how the so-called line-of-sight shear is systematically measurable from strong lensing images, paving the way for its use as a powerful new cosmological probe, which I have explained in more detail in my research statement.

My recent research experience in aspects of both weak and strong lensing, as well as my past experience in diverse areas of cosmology
, makes me an ideal candidate for the advertised role. I would also look forward to once again taking an active role in the wider cosmology community in Saclay, as I did while a postdoc at IPht, when I initiated and led a new series of joint journal clubs between the CosmoStat group and the cosmology group at IPhT, which continue to this day.

Thank you for your time and consideration, and I look forward to hearing from you in due course.s

\signature{Dr. Natalie Hogg}

\closing{Yours sincerely,}

%enclosure listing
%\encl{}

\end{letter}
\end{document}
