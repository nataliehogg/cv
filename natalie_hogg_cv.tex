\documentclass[11pt]{article} 
%https://www.latextemplates.com/template/wilson-resume-cv

% make accents work in tabbing (and then use \a'{e})
\usepackage{lmodern}
\usepackage[T1]{fontenc} 

%\usepackage[sorting=ydnt]{biblatex}

\usepackage{geometry}
\geometry{
	a4paper,
	total={170mm,257mm},
	left=20mm,
	right=20mm,
	bottom=20mm,
	top=20mm,
}

\usepackage{fontawesome}

\usepackage[super]{nth}

\usepackage{ebgaramond}

\usepackage{fancyhdr} % Customize the header and footer

\usepackage{lastpage} % Required for calculating the number of pages in the document

\usepackage[usenames,dvipsnames]{xcolor}

\usepackage{hyperref} % Colors for links, text and headings

\hypersetup{
	colorlinks,
	linkcolor={red!70!black},
	citecolor={red!70!black},
	urlcolor={red!70!black}
}

\setcounter{secnumdepth}{0} 
\definecolor{slateblue}{rgb}{0.17,0.22,0.34}

%\usepackage{sectsty} 
%\sectionfont{\color{slateblue}} 


\fancypagestyle{plain}{\fancyhf{}\cfoot{\thepage\ of \pageref{LastPage}}} % Define a custom page style
\pagestyle{plain} % Use the custom page style through the document
\renewcommand{\headrulewidth}{0pt} % Disable the default header rule
\renewcommand{\footrulewidth}{0pt} % Disable the default footer rule

\setlength\parindent{0pt} % Stop paragraph indentation

% Non-indenting itemize
\newenvironment{itemize-noindent}
{\setlength{\leftmargini}{0em}\begin{itemize}}
{\end{itemize}}

% Text width for tabbing environments
\newlength{\smallertextwidth}
\setlength{\smallertextwidth}{\textwidth}
\addtolength{\smallertextwidth}{-2cm}

\newcommand{\sqbullet}{~\vrule height 1ex width .8ex depth -.2ex} % Custom square bullet point definition

%----------------------------------------------------------------------------------------
%	MAIN HEADER COMMAND
%----------------------------------------------------------------------------------------

\renewcommand{\title}[1]{
%{\huge{\color{slateblue}\textbf{#1}}}\\ % Header section name and color
{\huge{\textbf{#1}}}\\
\rule{\textwidth}{0.3mm}\\ % Rule under the header
}

%----------------------------------------------------------------------------------------
%	JOB COMMAND
%----------------------------------------------------------------------------------------

\newcommand{\job}[6]{
\begin{tabbing}
\hspace{2cm} \= \kill
%\textbf{#1} \> \href{#4}{#3} \\
\textbf{#1} \> #3 \\
\textbf{#2} \>\+ \textit{#5} \\
\begin{minipage}{\smallertextwidth}
\vspace{2mm}
#6
\end{minipage}
\end{tabbing}
%\vspace{2mm}
}

%----------------------------------------------------------------------------------------
%	SKILL GROUP COMMAND
%----------------------------------------------------------------------------------------

\newcommand{\skillgroup}[2]{
\begin{tabbing}
\hspace{5mm} \= \kill
\sqbullet \>\+ \textbf{#1} \\
\begin{minipage}{\smallertextwidth}
\vspace{2mm}
#2
\end{minipage}
\end{tabbing}
}

%----------------------------------------------------------------------------------------
%	INTERESTS GROUP COMMAND
%-----------------------------------------------------------------------------------------

\newcommand{\interestsgroup}[1]{
\begin{tabbing}
\hspace{5mm} \= \kill
#1
\end{tabbing}
\vspace{-10mm}
}

\newcommand{\interest}[1]{\sqbullet \> \textbf{#1}\\[3pt]} % Define a custom command for individual interests

%----------------------------------------------------------------------------------------
%	TABBED BLOCK COMMAND
%----------------------------------------------------------------------------------------

\newcommand{\tabbedblock}[1]{
\begin{tabbing}
\hspace{2.5cm} \= \hspace{4cm} \= \kill
#1
\end{tabbing}
} 

\begin{document}

\title{Dr. Natalie B. Hogg} % Print the main header

\textbf{Postdoctoral researcher in cosmology at the \href{https://www.lupm.in2p3.fr/}{Laboratoire Univers et Particules}, Universit\'e de Montpellier}\\ \\
\textbf{Email:} \href{mailto:natalie.hogg@lupm.in2p3.fr}{natalie.hogg@lupm.in2p3.fr} | \textbf{Phone:} +33 7 66 72 42 18 | \textbf{Website:} \href{nataliebhogg.com}{nataliebhogg.com} | \textbf{Github:} \href{https://github.com/nataliehogg}{\faGithub} 

%\textbf{Area of research:} 12

%%----------------------------------------------------------------------------------------
%%	PERSONAL PROFILE
%%----------------------------------------------------------------------------------------
%
%\section{Personal Profile}
%
%Lorem ipsum dolor sit amet, consectetur adipiscing elit. Duis elementum nec dolor sed sagittis. Cras justo lorem, volutpat mattis lacus vel, consequat aliquam quam. Interdum et malesuada fames ac ante ipsum primis in faucibus. Integer blandit, massa at tincidunt ornare, dolor magna interdum felis, ac blandit urna neque in turpis.
%


%----------------------------------------------------------------------------------------
%	EMPLOYMENT HISTORY SECTION
%----------------------------------------------------------------------------------------

\section{Employment}

\job
{Oct 2025}{onwards}
{Institute of Astronomy, University of Cambridge}
{https://www.ast.cam.ac.uk/}
{Postdoctoral researcher}
{Developing likelihood-free inference techniques for strong lensing with Dr. Will Handley}

\job
{Oct 2023 --}{Oct 2025}
{Laboratoire univers et particules, Universit\a'{e} de Montpellier}
{https://www.lupm.in2p3.fr/}
{Postdoctoral researcher}
{Modelling strong lensing images for cosmology with Prof. Julien Larena.}

\job
{Feb 2022 --}{Oct 2023}
{Institut de physique th\a'{e}orique, Paris-Saclay}
{https://www.ipht.fr/en/}
{Postdoctoral researcher}
{Developing theoretical applications of strong lensing to cosmology, with Dr. Pierre Fleury.}

\job
{Jun 2021 --}{Feb 2022}
{Instituto de f\a'{i}sica t\a'{e}orica, Universidad Autonoma de Madrid}
{https://www.ift.uam-csic.es/en}
{Postdoctoral researcher}
{Working on applications of strong lensing and gravitational waves to cosmology, with Dr. Pierre Fleury and Dr. Matteo Martinelli.}

\section{Education}

\tabbedblock{
\bf{2017 --  2021} \> \textbf{PhD in Cosmology}, University of Portsmouth \\[5pt]
\> Thesis: \href{https://researchportal.port.ac.uk/portal/files/27089196/natalie_hogg_thesis.pdf}{Beyond $\Lambda$CDM: current and future constraints on alternative cosmological models} \\
\>\+ Supervisors: Dr. Marco Bruni, Prof. David Wands, Prof. Robert Crittenden
}

%------------------------------------------------

\tabbedblock{
\bf{2013 -- 2017} \> \textbf{MPhys Astrophysics \nth{1} class hons.}, Aberystwyth University\\
\>\+ with \textbf{Breen Prize} for best Master's dissertation in physics\\[5pt]
Master's dissertation: \href{https://nataliebhogg.files.wordpress.com/2017/06/dynamical_de_nbh.pdf}{Dynamical models of dark energy \& their background cosmological evolution}\\
Supervisor: Prof. Carsten van de Bruck (University of Sheffield; supervised remotely)
}

\section{Major awards}
\tabbedblock{
	\bf{2021} \> \textbf{G-Research quantitative research grant}: £2000\\
	\>\+ Hardship grant from major \href{https://www.gresearch.co.uk/article/our-may-grant-winners/}{financial technology company}
}
\tabbedblock{
	\bf{2017} \> \textbf{Breen Prize}: £2000\\
	\>\+ Best Master's dissertation in physics, Aberystwyth University
}

Plus \textbf{travel grants} awarded over the period 2017 -- 2020 from STFC, EU COST-Action CANTATA and Santander Bank totalling approximately £6000.

\section{Highlighted talks}
\textit{See Annex for a complete list of talks.}
\tabbedblock{
	\bf{Apr 2024} \> \textbf{Strong and weak lensing for cosmology}\\
	\>\+ Invited colloquium, ICG, University of Portsmouth
}

\tabbedblock{
	\bf{Aug 2023} \> \textbf{Strong and weak lensing for cosmology}\\
	\>\+ Invited plenary talk, ``Lensing at Different Scales'' workshop, University of Chicago
}
\tabbedblock{
	\bf{Apr 2023} \> \textbf{The weak lensing of strong lensing: a new cosmological probe} ~\href{https://astrotube.obspm.fr/w/ca0a828e-b3d8-45a9-ba63-f8ce33f3072e}{\faYoutubePlay}\\
	\>\+ Invited seminar, LUTH, Observatoire de Paris
}
\tabbedblock{
	\bf{Sep 2022} \> \textbf{Dancing in the dark: detecting a population of distant primordial black holes}\\
	\>\+ Invited seminar, IAS, Orsay
}
\tabbedblock{
	\bf{Apr 2022} \> \textbf{The distance duality relation: violations, constraints and biases}\\
	\>\+ Invited seminar, IAP, Paris
}
Since 2017, I have given 19 invited talks, contributed talks at 20 conferences and workshops, won 3 best talk prizes and given 5 outreach talks, including a Youtube video on the topic of standard sirens which has over 15,000 views ~\href{https://www.youtube.com/watch?v=DX2MiU2rX38}{\faYoutubePlay}.
%\\
%\textit{`Thanks ever so much for a fabulous presentation yesterday! It was incredible how you were able to present complex theories in such an accessible way.'} --- Feedback after a recent outreach talk on the history of general relativity and cosmology.
%----------------------------------------------------------------------------------------
%	IT/COMPUTING SKILLS SECTION
%----------------------------------------------------------------------------------------


\section{Supervision and teaching}

\tabbedblock{
	\bf{Supervision} \> Jason Makechemu, undergraduate summer research project on strong lensing, July \& August 2022. 
	\\
	\>\+ Jason is now a PhD student at the University of Lancaster supervised by Dr. Brooke Simmons.}

\tabbedblock{
	\bf{Mentorship} \> Th\a'{e}o Duboscq and Daniel Johnson, PhD students at the Universit\a'{e} de Montpellier (2023 -- present).
	\\
	\>\+ Marco Sebastianutti, PhD student at the University of Sussex (2022 -- 2023).}

\tabbedblock{
	\bf{Teaching} \> Lab demonstrator for Computational Physics course; coursework and exam marking for various\\
	\>\+ physics courses during my PhD (2017 -- 2019).}

\section{Open-source software development}

\skillgroup{\texttt{lenstronomy}~\href{https://github.com/lenstronomy}{\faGithub}}
{
\href{https://github.com/lenstronomy/lenstronomy/graphs/contributors?type=a}{\textbf{Top 4 contributor}} to this open-source Python package for strong gravitational lensing analysis.\\
$\rightarrow$ I led the implementation of a new  sub-package for line-of-sight effects. \\
$\rightarrow$ I implemented the \texttt{zeus} ensemble slice sampler for MCMC parameter inference.\\
$\rightarrow$ I initiated a restructuring of the MCMC sub-package to make it easier and more intuitive to use.
}

\skillgroup{\texttt{tdcosmo\_ext}~\href{https://github.com/nataliehogg/tdcosmo_ext}{\faGithub}}
{
Sole developer of this external strong lensing likelihood package for the Bayesian analysis software \texttt{Cobaya}.
}

%------------------------------------------------

\skillgroup{\texttt{analosis}~\href{https://github.com/nataliehogg/analosis}{\faGithub}}
{
Lead developer of this Python wrapper for \texttt{lenstronomy}, created to systematically generate and analyse large catalogues of strong lensing images.
}

%------------------------------------------------

\skillgroup{\texttt{darksirens}~\href{https://gitlab.com/matmartinelli/darksirens}{\faGitlab}}
{
Co-developer of this open-source Python package for creating mock gravitational wave event catalogues.
}

Experienced user of Python, Fortran, Bash, Git and HPC (SLURM, PBS). 

Other codes and software used includes \texttt{astropy}, \texttt{CAMB}, \texttt{CosmoMC}, \texttt{Cobaya}, \texttt{emcee}, \texttt{zeus}, \texttt{dynesty} and \texttt{Polychord}.

%----------------------------------------------------------------------------------------
%	INTERESTS SECTION
%----------------------------------------------------------------------------------------



\section{Leadership \& service}

\tabbedblock{
	\bf{Ongoing} \> \textbf{Referee} for MNRAS, JCAP, OJA, ApJ \& PDU.}

\tabbedblock{
	\bf{2024} \> \textbf{Organiser}, \href{https://beyondmainlens.sciencesconf.org/}{Strong Lensing Beyond the Main Lens workshop}, Montpellier.}

\tabbedblock{
	\bf{2022 -- 2023} \> \textbf{Organiser}, IPhT cosmology \& gravity journal club and IPhT cosmology seminar series.}

\tabbedblock{
	\bf{2020} \> \textbf{Chair}, local organising committee, \href{https://sites.google.com/port.ac.uk/south-coast-cosmology/home}{South Coast Cosmology meeting}, University of Portsmouth.}

%\tabbedblock{
%	\bf{2019} \> \textbf{Member}, local organising committee, \href{https://sites.google.com/port.ac.uk/frominfinitytozero}{A History of the Universe in Redshift} conference}

\tabbedblock{
	\bf{2018 -- 2019} \> \textbf{PhD student representative}, ICG management committee \& Faculty Research Degrees committee,\\ \>\+University of Portsmouth.}

%----------------------------------------------------------------------------------------

\section{\LARGE{Annex}}

\renewcommand\refname{Publications}
\bibliographystyle{unsrturl}
\nocite{*}
\bibliography{mypapers}

*PhD student at the time of publication whom I mentored. \\

\textbf{Bibliometrics}:
%\begin{itemize}
%	\item Total citations: 296
%	\item Mean citations: 26.9
%	\item h-index: 8
%\end{itemize}
Total citations: 296 --- Mean citations: 26.9 --- h-index: 8.
Numbers from \href{https://inspirehep.net/literature?sort=mostrecent&size=25&page=1&q=a%20n%20hogg&author_count=10%20authors%20or%20less&ui-citation-summary=true}{Inspire-HEP}, for published papers with fewer than ten authors.\\

I also contributed to the astrostatistics section of the 2021 \href{https://arxiv.org/abs/2110.10074}{EUCAPT} White Paper, and am a signatory of the Snowmass 2021 \href{https://arxiv.org/abs/2203.06142}{Cosmology Intertwined} White Paper.

\section{Talks}
\tabbedblock{
	\bf{Dec 2024} \> \textbf{The weak lensing of strong lensing: a new probe of cosmology}\\
	\>\+ Invited seminar, APC, Paris (Cité)
}
\tabbedblock{
	\bf{Dec 2024} \> \textbf{The weak lensing of strong lensing: a new probe of cosmology}\\
	\>\+ Invited seminar, LPNHE, Paris (Sorbonne)
}

\tabbedblock{
	\bf{Nov 2024} \> \textbf{Lensing for cosmology: theory review}\\
	\>\+ Invited plenary talk, News from the Dark, Marseille
}

\tabbedblock{
	\bf{Apr 2024} \> \textbf{Strong and weak lensing for cosmology}\\
	\>\+ Invited colloquium, ICG, University of Portsmouth
}

\tabbedblock{
	\bf{Apr 2024} \> \textbf{Strong lensing constraints on dark energy}\\
	\>\+ Invited seminar, CPPM, Marseille
}

\tabbedblock{
	\bf{Aug 2023} \> \textbf{Strong and weak lensing for cosmology}\\
	\>\+ Invited plenary talk, ``Lensing at Different Scales'' workshop, University of Chicago
}

\tabbedblock{
	\bf{Jun 2023} \> \textbf{Cosmology’s first century: a guided tour of the Universe from Einstein to JWST
	}\\
	\>\+ Invited outreach talk, Charlbury Beer Festival
}

\tabbedblock{
	\bf{May 2023} \> \textbf{The weak lensing of strong lensing: a new cosmological probe}\\
	\>\+ Contributed talk, Progress on Old and New Themes in Cosmology conference, Avignon
}

\tabbedblock{
	\bf{Apr 2023} \> \textbf{The weak lensing of strong lensing: a new cosmological probe} ~\href{https://astrotube.obspm.fr/w/ca0a828e-b3d8-45a9-ba63-f8ce33f3072e}{\faYoutubePlay}\\
	\>\+ Invited seminar, LUTH, Observatoire de Paris
}

\tabbedblock{
	\bf{Nov 2022} \> \textbf{One Ring to Rule Them All: line-of-sight shear as a new cosmological probe
	}\\
	\>\+ Contributed talk, Recontres des jeunes physicien.nes, Coll\a`{e}ge de France
}

\tabbedblock{
	\bf{Oct 2022} \> \textbf{Dancing in the dark: detecting a population of distant primordial black holes}\\
	\>\+ Invited seminar, ICG, University of Portsmouth
}

\tabbedblock{
	\bf{Sep 2022} \> \textbf{Dancing in the dark: detecting a population of distant primordial black holes}\\
	\>\+ Invited seminar, IAS, Orsay
}

\tabbedblock{
	\bf{Jun 2022} \> \textbf{Measuring the LOS shear using Einstein rings: a proof-of-concept
	}\\
	\>\+ Contributed talk, Line-of-sight effects in strong gravitational lensing workshop, Montpellier
}

\tabbedblock{
	\bf{Jun 2022} \> \textbf{Dancing in the dark: detecting a population of distant primordial black holes}\\
	\>\+ Invited talk, THC meeting, Leiden
}

\tabbedblock{
	\bf{May 2022} \> \textbf{Understanding the dark Universe}\\
	\>\+ Contributed talk, IPhT retreat, Autrans
}

\tabbedblock{
	\bf{May 2022} \> \textbf{The curious incident of the distances in the nighttime}\\
	\>\+ Department seminar, IPhT
}

\tabbedblock{
	\bf{May 2022} \> \textbf{The resilience of the Etherington--Hubble relation}\\
	\>\+ Contributed talk, Action Dark Energy meeting, Marseille
}

\tabbedblock{
	\bf{May 2022} \> \textbf{Shan--Chen interacting vacuum cosmology}\\
	\>\+ Contributed talk, Action Dark Energy meeting, Marseille
}

\tabbedblock{
	\bf{Apr 2022} \> \textbf{The distance duality relation: violations, constraints and biases}\\
	\>\+ Invited seminar (``GRECO''), IAP, Paris
}

\tabbedblock{
	\bf{Apr 2022} \> \textbf{The distance duality relation: violations, constraints and biases}~\href{https://youtu.be/yu4bgxIuVFM?si=-v6lrjn8pZr9Vc_H}{\faYoutubePlay}\\
	\>\+ Invited seminar, IFT, Madrid (\textit{delivered remotely})
}
\tabbedblock{
	\bf{Oct 2021} \> \textbf{The distance duality relation: violations, constraints and biases}\\
	\>\+ Invited seminar (``CAPT''), University of Nottingham (\textit{delivered remotely})
}
\tabbedblock{
	\bf{Jul 2021} \> \textbf{Novel probes of the distance duality relation}\\
	\>\+ Invited seminar, University of Leiden (\textit{delivered remotely})
}
\tabbedblock{
	\bf{Jul 2021} \> \textbf{Constraints on the distance duality relation with standard sirens} ~\href{https://youtu.be/fqtZTJjgotQ?si=Z9TsQfzdc8L_AseX}{\faYoutubePlay}\\
	\>\+ Contributed talk, Cosmology from Home (\textit{delivered remotely})
}
\tabbedblock{
	\bf{May 2021} \> \textbf{Constraints on the distance duality relation with standard sirens}\\
	\>\+ Contributed talk, EUCAPT Symposium, CERN (\textit{delivered remotely})
}
\tabbedblock{
	\bf{Apr 2021} \> \textbf{Is the standard model of cosmology wrong?}\\
	\>\+ Invited seminar, Aberystwyth University (\textit{delivered remotely})
}
\tabbedblock{
	\bf{Apr 2021} \> \textbf{Constraints on the distance duality relation with standard sirens}\\
	\>\+ Contributed talk, Britgrav, University College Dublin (\textit{delivered remotely})
}
\tabbedblock{
	\bf{Jun 2020} \> \textbf{Standard sirens and the distance duality relation}\\
	\>\+ Invited seminar, ICG, University of Portsmouth (\textit{delivered remotely})
}
\tabbedblock{
	\bf{Apr 2020} \> \textbf{New constraints on beyond-$\Lambda$CDM cosmologies}\\
	\>\+ Invited seminar, Queen Mary University of London (\textit{delivered remotely})
}

\tabbedblock{
	\bf{Dec 2019} \> \textbf{Interacting vacuum dark energy}\\
	\>\+ Invited seminar (``CRAG''), University of Sheffield 
}

\tabbedblock{
	\bf{Dec 2019} \> \faTrophy~\textbf{Interacting vacuum dark energy}\\
	\>\+ Contributed talk, TEXAS Symposium, Portsmouth 
}

\tabbedblock{
	\bf{Jun 2019} \> \faTrophy~\textbf{Testing the standard model of cosmology}\\
	\>\+ Contributed talk, Faculty of Technology Research \& Innovation conference, University of Portsmouth 
}

\tabbedblock{
	\bf{Mar 2019} \> \textbf{Is the standard model of cosmology wrong?}\\
	\>\+ Contributed talk, 26eme Congr\a'{e}s des Doctorants, Paris
}

\tabbedblock{
	\bf{Jan 2019} \> \textbf{Interacting vacuum dark energy}\\
	\>\+ Department seminar, ICG, University of Portsmouth
}

\tabbedblock{
	\bf{Jan 2019} \> \textbf{New constraints on interacting vacuum dark energy}\\
	\>\+ Contributed talk, South Coast Cosmology meeting, University of Portsmouth
}

\tabbedblock{
	\bf{Oct 2018} \> \textbf{Interacting vacuum dark energy}\\
	\>\+ Contributed talk, CANTATA meeting, Valencia
}

\tabbedblock{
	\bf{May 2018} \> \faTrophy~\textbf{Dark energy: theories and observations}\\
	\>\+ Contributed talk, British Federation of Women Graduates Annual meeting, London
}
\tabbedblock{
	\bf{Apr 2018} \> \textbf{Interacting vacuum dark energy}\\
	\>\+ Contributed talk, Britgrav, University of Portsmouth
}
\tabbedblock{
	\bf{Apr 2018} \> \textbf{Interacting vacuum dark energy}\\
	\>\+ Contributed talk, Advances in HEP and Cosmology, University of Southampton
}

\tabbedblock{
	\bf{Dec 2017} \> \textbf{Interacting dark energy}\\
	\>\+ Contributed talk, Euclid UK Consortium Meeting, University of Portsmouth
}

\tabbedblock{
	\bf{Nov 2017} \> \textbf{Dynamical dark energy}\\
	\>\+ Contributed talk, PhD student day, University of Portsmouth
}
\tabbedblock{
	\bf{Sep 2017} \> \textbf{Dynamical dark energy}\\
	\>\+ Contributed talk, CANTATA Cosmology Summer School, Corfu
}

\faTrophy~ indicates best talk prize.
%----------------------------------------------------------------------------------------
%	REFEREE SECTION
%----------------------------------------------------------------------------------------

%\section{References}
%
%\begin{tabbing}
%\hspace{4.4cm} \= \hspace{4cm} \= \kill % Spacing within the block
%{Prof. Julien Larena} \> \href{mailto:julien.larena@umontpellier.fr}{julien.larena@umontpellier.fr} \\ % Referee name
%{Dr. Pierre Fleury} \> \href{mailto:pierre.fleury@lupm.in2p3.fr}{pierre.fleury@lupm.in2p3.fr} \\ % Referee company
%{Prof. Simon Birrer} \> \href{mailto:simon.birrer@stonybrook.edu}{simon.birrer@stonybrook.edu} \\ % Referee job title 
%%{Prof. Carsten van de Bruck} \> \href{mailto:c.vandebruck@sheffield.ac.uk}{c.vandebruck@sheffield.ac.uk} % Referee contact information
%\end{tabbing}


%----------------------------------------------------------------------------------------

\end{document}